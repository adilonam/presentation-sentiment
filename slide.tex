\documentclass{beamer}
\usepackage{hyperref}
\usepackage[T1]{fontenc}
\usepackage{tikz}
\usepackage{xcolor}
% A SISU beamer based on THU beamer.

% other packages
\usepackage{latexsym,amsmath,xcolor,multicol,booktabs,calligra}
\usepackage{graphicx,pstricks,listings,stackengine}
\renewcommand{\figurename}{Figure.}
\renewcommand{\tablename}{Table.}


\definecolor{appbg}{HTML}{f0f0f0}
\definecolor{chatbg}{HTML}{ffffff}
\definecolor{headerbg}{HTML}{4CAF50}
\definecolor{inputbg}{HTML}{e0e0e0}
\definecolor{sendbtn}{HTML}{4CAF50}
\definecolor{navbarbg}{HTML}{333333}
\definecolor{navbartext}{HTML}{ffffff}
\definecolor{windowborder}{HTML}{999999}



\author{Adil ABBADI , Abdelhak MEKAOUI}
\title{Conception, Développement et Déploiement d'un Chatbot Inte lligent et d'une Plate forme E-Learning}
\institute{ENSA Agadir}
\date{\today}
\usepackage{sisu}

% defs
\def\cmd#1{\texttt{\color{red}\footnotesize $\backslash$#1}}
\def\env#1{\texttt{\color{blue}\footnotesize #1}}
\definecolor{deepblue}{rgb}{0,0,0.5}
\definecolor{deepred}{rgb}{0.6,0,0}
\definecolor{deepgreen}{rgb}{0,0.5,0}
\definecolor{halfgray}{gray}{0.55}

\lstset{
    basicstyle=\ttfamily\small,
    keywordstyle=\bfseries\color{deepblue},
    emphstyle=\ttfamily\color{deepred},    % Custom highlighting style
    stringstyle=\color{deepgreen},
    numbers=left,
    numberstyle=\small\color{halfgray},
    rulesepcolor=\color{red!20!green!20!blue!20},
    frame=shadowbox,
}


\begin{document}




\bibliographystyle{plain}
\bibliography{mybibfile}


\begin{frame}

  \begin{figure}[htpb]
        \begin{center}
         \includegraphics[width=0.45\linewidth]{pic/logo.png}
          \hspace{0.05\linewidth}
            \includegraphics[width=0.45\linewidth]{pic/code.png}
        \end{center}
    \end{figure}
    
    \titlepage
  
\end{frame}


\begin{frame}
    \tableofcontents[sectionstyle=show,subsectionstyle=show/shaded/hide,subsubsectionstyle=show/shaded/hide]
\end{frame}

\section{Introduction}

\subsection{Contexte general}
\begin{frame}{Contexte general}



 
    \vspace{0.5cm} % Adjust space between image and text
    \begin{block}{\centering \textbf{\Large Le centre Code 212}}
        \centering
        \vspace{0.2cm} % Adjust space within the block
        Le Code 212, lancé dans le cadre du PACTE ESRI-2030 au Maroc, est un centre de formation et de certification dédié aux métiers du digital, visant à renforcer les compétences numériques et à promouvoir l’innovation pour répondre aux besoins du marché.
    \end{block}

\begin{figure}[htpb]
        \centering
        \includegraphics[width=0.45\linewidth]{pic/code.png}
    \end{figure}
\end{frame}


\subsection{Problematique}
\begin{frame}{Problematique}
\begin{figure}[htpb]
        \centering
        \includegraphics[height=3cm]{pic/question.png}
    \end{figure}
   Avec la digitalisation croissante, Code 212 adapte les outils éducatifs pour offrir une interaction personnalisée et immédiate via un chatbot IA, répondant ainsi aux besoins actuels des étudiants.
\end{frame}

\subsection{Solution proposee}
\begin{frame}{Solution proposee}

Le chatbot IA de Code 212 offre une assistance instantanée aux étudiants en temps réel, optimisant ainsi les ressources pédagogiques du centre.
\begin{figure}[htpb]
        \centering
        \includegraphics[height=3cm]{pic/ia.png}
    \end{figure}
\end{frame}



\subsection{Fonctionnalités attendues}
\begin{frame}{Fonctionnalités attendues}

\begin{itemize}
    \setlength\itemsep{0.8em} % Adjust the spacing between items
     \item \textbf{Interaction avec chatbot:} repondre aux questions des etudiants en tempas reel.
    \item \textbf{Inscription à un cours:} Inscription rapide en ligne
    \item \textbf{Inscription à un événement:} Participation facile aux événements
    \item \textbf{Demande de certificat:} Obtention de certificats via chatbot
    \item \textbf{Gestion des cours:} Administration des cours par gestionnaire
    \item \textbf{Gestion des événements:} Organisation des événements
    \item \textbf{Gestion des certificats:} Administration des certificats
    \item \textbf{Gestion des utilisateurs:} Administration des utilisateurs
\end{itemize}

\end{frame}






\section{Conception du Project}
\subsection{Description du Project}
\begin{frame}{Description du Project}
\begin{figure}[htpb]
        \centering
        \includegraphics[height=5cm]{pic/Code212_architecture.drawio (1).png}
    \end{figure}

\end{frame}

\subsection{Digramme cas d'utilisation}
\begin{frame}{Digramme cas d'utilisation}
 \begin{figure}[htpb]
        \centering
        \includegraphics[height=7.5cm]{pic/usecase.png}
    \end{figure}
\end{frame}

\subsection{Planification}
\begin{frame}{Planification}
 
 \begin{figure}[htpb]
        \centering
        \includegraphics[height=7.5cm]{pic/gantt.png}
    \end{figure}
\end{frame}


\subsection{Prototype}
\begin{frame}{Prototype}

 \begin{figure}[htpb]
        \centering
        \includegraphics[height=6cm]{pic/prototype.png}
    \end{figure}
    
\end{frame}


\section{Spécifications Techniques}



\subsection{Architecture microsevice}



\begin{frame}{Architecture microservice}
   \begin{figure}[htpb]
        \centering
        \includegraphics[height=5cm]{pic/Code212_architecture.drawio (1).png}
    \end{figure}
\end{frame}


\subsection{Benchmark}
\begin{frame}{Benchmark}
   \begin{figure}[htpb]
        \centering
        \includegraphics[height=5cm]{pic/benchmark.png}
    \end{figure}
\end{frame}


\subsection{Technologies Sélectionnées}
\begin{frame}{Technologies Sélectionnées}
    \begin{figure}[htpb]
        \centering
        \begin{minipage}{0.32\textwidth}
            \centering
            \includegraphics[height=3cm]{pic/next.png}
        \end{minipage}%
        \hspace{0.03\textwidth}
        \begin{minipage}{0.32\textwidth}
            \centering
            \includegraphics[height=3cm]{pic/spring.png}
        \end{minipage}%
        \hspace{0.03\textwidth}
        \begin{minipage}{0.32\textwidth}
            \centering
            \includegraphics[height=4cm]{pic/keycloak.png}
        \end{minipage}
         \hspace{0.03\textwidth}
        \begin{minipage}{0.32\textwidth}
            \centering
            \includegraphics[height=4cm]{pic/ia.png}
        \end{minipage}
    \end{figure}
\end{frame}

\subsection{Communication entre services}
\begin{frame}{Communication entre services}
   \begin{figure}[htpb]
        \centering
        \includegraphics[height=2cm]{pic/rest.png}
         \hspace{0.1\textwidth}
         \includegraphics[height=2cm]{pic/kafka.png}
    \end{figure}
\end{frame}

\subsection{Conclusion}
\begin{frame}{Conclusion}
 
\end{frame}

\section{Project Management}

\subsection{Methodology}
\begin{frame}{Methodology}
    Afin de gérer au mieux ce projet, nous avons opté pour la méthode SCRUM et divisé les différents modules en sprints.
\end{frame}

\subsection{Sprints}
\begin{frame}{Sprints}
    Un sprint de préparation a été initié pour effectuer une analyse fonctionnelle globale et construire un modèle de base pour les sprints suivants.
\end{frame}

\subsection{Execution}
\begin{frame}{Execution}
    Après avoir terminé le sprint de préparation, nous avons exécuté les autres sprints, chacun étant dédié à la conception détaillée d’un module de l’application, suivi par l’élaboration de ses maquettes et enfin sa réalisation.
\end{frame}

\section{Conclusion}

\begin{frame}{Conclusion}
    En résumé, le projet vise à intégrer un chatbot AI dans une plateforme e-learning, offrant une expérience d’apprentissage améliorée pour les étudiants et une gestion efficace des cours et événements pour les gestionnaires et administrateurs.
\end{frame}

\section{References}

\begin{frame}[allowframebreaks]
    \bibliography{ref}
    \bibliographystyle{alpha}
\end{frame}


\end{document}
