\section{Introduction}

\subsection{Contexte general}


\begin{frame}{Organisme d'accueil}

    \vspace{0.5cm} % Adjust space between image and text
    \begin{block}{\centering \textbf{\Large Le centre Code 212}}
        \centering
        \vspace{0.2cm} % Adjust space within the block
        Le Code 212, lancé dans le cadre du PACTE ESRI-2030 au Maroc, est un centre de formation et de certification dédié aux métiers du digital, visant à renforcer les compétences numériques et à promouvoir l’innovation pour répondre aux besoins du marché.
    \end{block}

    \begin{figure}[htpb]
        \centering
        \begin{minipage}[b]{0.45\linewidth}
            \centering
            \includegraphics[width=\linewidth]{assets/images/esri.png}
        \end{minipage}
        \begin{minipage}[b]{0.45\linewidth}
            \centering
            \includegraphics[width=\linewidth]{assets/images/code.png}
        \end{minipage}
    \end{figure}
\end{frame}


\begin{frame}{Centre Code212 : Ses Fonctionnalités et Services}

    \begin{figure}[htpb]
        \centering
        \begin{minipage}[b]{0.45\linewidth}
            \centering
            \includegraphics[width=\linewidth]{assets/images/formation.jpg}
        \end{minipage}
        \begin{minipage}[b]{0.45\linewidth}
            \centering
            \includegraphics[width=\linewidth]{assets/images/certificat.png}
        \end{minipage}
    \end{figure}
\end{frame}



\subsection{Problematique}
\begin{frame}{Problematique}
    \begin{figure}[htpb]
        \centering
        \includegraphics[height=3cm]{assets/images/question.png}
    \end{figure}
    Face à la diversité des services offertes par Code 212 et l'université UIZ en général, les étudiants sont confrontés à une multitude de questions en temps réel. Quelle solution technologique peut répondre efficacement à ce besoin en fournissant un guide instantané et personnalisé ?
\end{frame}

\subsection{Solution proposee}
\begin{frame}{Solution proposee}

    Le chatbot IA de Code 212 offre une assistance instantanée aux étudiants en temps réel, optimisant ainsi les ressources pédagogiques du centre.
    \begin{figure}[htpb]
        \centering
        \includegraphics[height=3cm]{assets/images/ia.png}
    \end{figure}
\end{frame}

\begin{frame}{Un Chatbot Inspiré par ChatGPT}

    S\'inspirant de ChatGPT, un outil de réponse automatisée, une solution similaire pourrait efficacement répondre aux besoins de recherche des utilisateurs, en fournissant des informations pertinentes en temps réel pour faciliter l\'accès et le traitement des connaissances.

    \begin{figure}[htpb]
        \centering
        \includegraphics[height=3cm]{assets/images/chatgpt.png}
    \end{figure}
\end{frame}

\subsection{Fonctionnalités attendues}


\begin{frame}{Fonctionnalités Axées sur l'Utilisateur}
    \begin{figure}[htpb]
        \centering
        \includegraphics[height=3cm]{assets/images/user.png}
    \end{figure}
    \begin{itemize}
        \setlength\itemsep{0.8em} % Adjust the spacing between items
        \item \textbf{Interaction avec chatbot}
        \item \textbf{Inscription à un cours}
        \item \textbf{Inscription à un événement}
        \item \textbf{Demande de certificat}

    \end{itemize}

\end{frame}


\begin{frame}{Fonctionnalités Axées sur l'administrateur}
    \begin{figure}[htpb]
        \centering
        \includegraphics[height=3cm]{assets/images/admin.png}
    \end{figure}

    \begin{itemize}
        \setlength\itemsep{0.8em} % Adjust the spacing between items

        \item \textbf{Gestion des cours}
        \item \textbf{Gestion des événements}
        \item \textbf{Gestion des certificats}
        \item \textbf{Gestion des utilisateurs}
    \end{itemize}

\end{frame}
