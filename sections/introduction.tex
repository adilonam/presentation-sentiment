\section{Introduction}

\subsection{Contexte general}


\begin{frame}{Organisme d'accueil}

    \vspace{0.5cm} % Adjust space between image and text
    \begin{block}{\centering \textbf{\Large Le centre Code 212}}
        \centering
        \vspace{0.2cm} % Adjust space within the block
        Code 212 : Hub d'excellence numérique du PACTE ESRI-2030. Centre de compétences spécialisé en transformation digitale, offrant certification professionnelle et solutions innovantes alignées sur les exigences du marché.
    \end{block}

    \begin{figure}[H]
        \centering
        \begin{minipage}[b]{0.45\linewidth}
            \centering
            \includegraphics[width=\linewidth]{assets/images/esri.png}
        \end{minipage}
        \begin{minipage}[b]{0.45\linewidth}
            \centering
            \includegraphics[width=\linewidth]{assets/images/code.png}
        \end{minipage}
    \end{figure}
\end{frame}


\begin{frame}{Centre Code212 : Ses Fonctionnalités et Services}

    \begin{figure}[H]
        \centering
        \begin{minipage}[b]{0.45\linewidth}
            \centering
            \includegraphics[width=\linewidth]{assets/images/formation.jpg}
        \end{minipage}
        \begin{minipage}[b]{0.45\linewidth}
            \centering
            \includegraphics[width=\linewidth]{assets/images/certificat.png}
        \end{minipage}
    \end{figure}
\end{frame}



\subsection{Problématique}
\begin{frame}{Problématique}
    \begin{figure}[H]
        \centering
        \includegraphics[height=3cm]{assets/images/question.png}
    \end{figure}
    
    \vspace{0.5cm}
    
    \begin{block}{\centering \textbf{Défis Organisationnels}}
        \begin{itemize}
            \setlength\itemsep{0.6em}
            \item \textbf{Média numérique} : Dispersion des canaux d'information
            \item \textbf{Surcharge informationnelle} : Volume excessif de données à traiter
            \item \textbf{Process rapide} : Besoin d'accélération des workflows
        \end{itemize}
    \end{block}
    
    \vspace{0.3cm}
    
    \begin{alertblock}{\centering \textbf{Problématique}}
        \centering
        Comment automatiser l'accès à l'information et optimiser les processus métier face à la prolifération des données numériques ?
    \end{alertblock}
\end{frame}

\subsection{Solution proposee}
\begin{frame}{Solution proposee}

    Le chatbot IA de Code 212 offre une assistance instantanée aux étudiants en temps réel, optimisant ainsi les ressources pédagogiques du centre.
    \begin{figure}[H]
        \centering
        \includegraphics[height=3cm]{assets/images/ia.png}
    \end{figure}
\end{frame}

\begin{frame}{Un Chatbot Inspiré par ChatGPT}

    S\'inspirant de ChatGPT, un outil de réponse automatisée, une solution similaire pourrait efficacement répondre aux besoins de recherche des utilisateurs, en fournissant des informations pertinentes en temps réel pour faciliter l\'accès et le traitement des connaissances.

    \begin{figure}[H]
        \centering
        \includegraphics[height=3cm]{assets/images/chatgpt.png}
    \end{figure}
\end{frame}

\subsection{Fonctionnalités attendues}


\begin{frame}{Fonctionnalités Axées sur l'Utilisateur}
    \begin{figure}[H]
        \centering
        \includegraphics[height=3cm]{assets/images/user.png}
    \end{figure}
    \begin{itemize}
        \setlength\itemsep{0.8em} % Adjust the spacing between items
        \item \textbf{Interaction avec chatbot}
        \item \textbf{Inscription à un cours}
        \item \textbf{Inscription à un événement}
        \item \textbf{Demande de certificat}

    \end{itemize}

\end{frame}


\begin{frame}{Fonctionnalités Axées sur l'administrateur}
    \begin{figure}[H]
        \centering
        \includegraphics[height=3cm]{assets/images/admin.png}
    \end{figure}

    \begin{itemize}
        \setlength\itemsep{0.8em} % Adjust the spacing between items

        \item \textbf{Gestion des cours}
        \item \textbf{Gestion des événements}
        \item \textbf{Gestion des certificats}
        \item \textbf{Gestion des utilisateurs}
    \end{itemize}

\end{frame}
