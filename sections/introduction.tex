\section{Introduction}

\subsection{Contexte general}

\begin{frame}{Organisme d'accueil}

    \vspace{0.5cm} % Adjust space between image and text
    \begin{block}{\centering \textbf{\Large Le centre Code 212}}
        \centering
        \vspace{0.2cm} % Adjust space within the block
        Code 212 : Hub d'excellence numérique du PACTE ESRI-2030. Centre de compétences spécialisé en transformation digitale, offrant certification professionnelle et solutions innovantes alignées sur les exigences du marché.
    \end{block}

    \begin{figure}[H]
        \centering
        \begin{minipage}[b]{0.45\linewidth}
            \centering
            
            \includegraphics[width=\linewidth]{assets/images/esri.png}
        \end{minipage}
        \begin{minipage}[b]{0.45\linewidth}
            \centering
            \includegraphics[width=\linewidth]{assets/images/code.png}
        \end{minipage}
    \end{figure}
\end{frame}


\begin{frame}{Centre Code212 : Ses Fonctionnalités et Services}

    \begin{figure}[H]
        \centering
        \begin{minipage}[b]{0.45\linewidth}
            \centering
            \includegraphics[width=\linewidth]{assets/images/formation.jpg}
        \end{minipage}
        \begin{minipage}[b]{0.45\linewidth}
            \centering
            \includegraphics[width=\linewidth]{assets/images/certificat.png}
        \end{minipage}
    \end{figure}
\end{frame}

\subsection{Fondements théoriques du sentiment}

\begin{frame}{Qu'est-ce que le sentiment ?}
    \begin{block}{\centering \textbf{Définition psychologique}}
        \begin{itemize}
            \setlength\itemsep{0.6em}
            \item \textbf{État affectif durable} d'intensité modérée
            \item \textbf{Dirigé vers un objet} ou une situation
            \item \textbf{Composante cognitive} importante
            \item Se distingue des \textbf{émotions brèves}
        \end{itemize}
    \end{block}
    
    \vspace{0.3cm}
    
    \begin{exampleblock}{\centering \textbf{Exemples}}
        \centering
        Amour, haine, confiance, jalousie : états affectifs \textbf{durables} et \textbf{complexes}
    \end{exampleblock}
\end{frame}

\begin{frame}{Émotions vs Sentiments}
    \begin{columns}
        \begin{column}{0.48\textwidth}
            \begin{block}{\centering \textbf{Émotions}}
                \begin{itemize}
                    \setlength\itemsep{0.4em}
                    \item \textbf{Brèves} (secondes/minutes)
                    \item \textbf{Intenses}
                    \item Changements \textbf{physiologiques}
                    \item Réponse à un \textbf{stimulus}
                    \item 6 émotions de base (Ekman)
                \end{itemize}
            \end{block}
        \end{column}
        \begin{column}{0.48\textwidth}
            \begin{block}{\centering \textbf{Sentiments}}
                \begin{itemize}
                    \setlength\itemsep{0.4em}
                    \item \textbf{Durables} (heures/jours)
                    \item \textbf{Modérés}
                    \item Composante \textbf{cognitive}
                    \item État \textbf{dispositionnel}
                    \item Dirigés vers un \textbf{objet}
                \end{itemize}
            \end{block}
        \end{column}
    \end{columns}
    
    \vspace{0.3cm}
    
    \begin{alertblock}{\centering \textbf{Continuité}}
        \centering
        Des émotions répétées peuvent alimenter un sentiment durable
    \end{alertblock}
\end{frame}

\begin{frame}{Typologie des sentiments}
    \begin{block}{\centering \textbf{Valence émotionnelle}}
        \begin{itemize}
            \setlength\itemsep{0.6em}
            \item \textbf{Positif} : satisfaction, enthousiasme, joie
            \item \textbf{Négatif} : mécontentement, colère, tristesse  
            \item \textbf{Neutre} : absence de position affective tranchée
            \item \textbf{Mixte} : sentiments ambivalents (positif + négatif)
        \end{itemize}
    \end{block}
    

    
\end{frame}

\begin{frame}{Historique de l'analyse de sentiment}
    \begin{block}{\centering \textbf{Chronologie}}
        \begin{itemize}
            \setlength\itemsep{0.5em}
            \item \textbf{2001-2002} : Travaux fondateurs (Das \& Chen, Pang et al.)
            \item \textbf{2003} : Émergence des termes "sentiment analysis" et "opinion mining"
            \item \textbf{2008} : Première synthèse majeure (Pang \& Lee)
            \item \textbf{2010s} : Essor de l'apprentissage profond (LSTM, CNN)
            \item \textbf{2018+} : Révolution des Transformers (BERT, GPT)
        \end{itemize}
    \end{block}
    
    \vspace{0.3cm}
    
    \begin{alertblock}{\centering \textbf{Moteur}}
        \centering
        Explosion du Web 2.0 et des contenus générés par les utilisateurs
    \end{alertblock}
\end{frame}

\begin{frame}{Méthodes d'analyse de sentiment}
    \begin{block}{\centering \textbf{Approches classiques}}
        \begin{itemize}
            \setlength\itemsep{0.4em}
            \item \textbf{Lexiques de sentiments} : listes de mots annotés
            \item \textbf{Machine Learning} : SVM, Naive Bayes, Régression logistique
        \end{itemize}
    \end{block}
    
    \begin{block}{\centering \textbf{Apprentissage profond}}
        \begin{itemize}
            \setlength\itemsep{0.4em}
            \item \textbf{CNN} : Réseaux de neurones convolutifs
            \item \textbf{LSTM/BiLSTM} : Mémoire à long/court terme
        \end{itemize}
    \end{block}
    
    \begin{block}{\centering \textbf{Modèles Transformer}}
        \begin{itemize}
            \setlength\itemsep{0.4em}
            \item \textbf{BERT, T5, GPT} : Embeddings contextuels
            \item \textbf{Mécanismes d'attention} : Performances état de l'art
        \end{itemize}
    \end{block}
\end{frame}

\begin{frame}{Défis actuels}
    \begin{alertblock}{\centering \textbf{Complexités linguistiques}}
        \begin{itemize}
            \setlength\itemsep{0.5em}
            \item \textbf{Ironie et sarcasme} : Inversion du sentiment apparent
            \item \textbf{Sentiments mixtes} : Ambivalence positive-négative
            \item \textbf{Contexte culturel} : Variations selon les langues
            \item \textbf{Évolution du langage} : Argot, émojis, néologismes
        \end{itemize}
    \end{alertblock}

    \vspace{0.3cm}
    
    \begin{block}{\centering \textbf{Extensions multimodales}}
        \centering
        Intégration du ton vocal, expressions faciales, gestuelle
    \end{block}
\end{frame}



\subsection{Problématique}
\begin{frame}{Problématique}
    \begin{figure}[H]
        \centering
        \includegraphics[height=3cm]{assets/images/question.png}
    \end{figure}
    
    \vspace{0.5cm}
    
    \begin{block}{\centering \textbf{Défis de l'Analyse de Sentiments}}
        \begin{itemize}
            \setlength\itemsep{0.6em}
            \item \textbf{Volume de données} : Milliers de commentaires quotidiens sur Hespress
            \item \textbf{Analyse manuelle} : Processus chronophage et subjectif
            \item \textbf{Temps réel} : Besoin d'analyse instantanée des sentiments
        \end{itemize}
    \end{block}
    
    \vspace{0.3cm}
    
    \begin{alertblock}{\centering \textbf{Problématique}}
        \centering
        Comment analyser efficacement les sentiments exprimés dans les commentaires des actualités Hespress pour comprendre l'opinion publique marocaine en temps réel ?
    \end{alertblock}
\end{frame}

\subsection{Solution proposee}
\begin{frame}{Solution proposee}
    Application de classification de sentiments des commentaires Hespress : analyse automatisée en temps réel de l'opinion publique via intelligence artificielle et architecture microservices moderne.
    \begin{figure}[H]
        \centering
        \includegraphics[height=3cm]{assets/images/ia.png}
    \end{figure}
\end{frame}

\begin{frame}{Architecture Technique}
    \begin{block}{\centering \textbf{Stack Technologique}}
        \begin{itemize}
            \setlength\itemsep{0.6em}
            \item \textbf{Frontend} : Next.js pour une interface moderne
            \item \textbf{Backend} : FastAPI pour des API performantes
            \item \textbf{Base de données} : PostgreSQL avec cache Redis
            \item \textbf{Authentification} : Keycloak pour la sécurité
            \item \textbf{Scraping} : Selenium pour l'extraction de données
            \item \textbf{IA} : Modèle cardiffnlp/twitter-xlm-roberta-base-sentiment
            \item \textbf{API Gateway} : Spring pour l'orchestration
        \end{itemize}
    \end{block}
\end{frame}

\subsection{Fonctionnalités attendues}

\begin{frame}{Fonctionnalités Axées sur l'Utilisateur}
    \begin{figure}[H]
        \centering
        \includegraphics[height=3cm]{assets/images/user.png}
    \end{figure}
    \begin{itemize}
        \setlength\itemsep{0.8em} % Adjust the spacing between items
        \item \textbf{Visualisation des sentiments en temps réel}
        \item \textbf{Recherche et filtrage par article}
        \item \textbf{Statistiques et graphiques d'analyse}
        \item \textbf{Export des données d'analyse}
    \end{itemize}
\end{frame}

\begin{frame}{Fonctionnalités Axées sur l'administrateur}
    \begin{figure}[H]
        \centering
        \includegraphics[height=3cm]{assets/images/admin.png}
    \end{figure}

    \begin{itemize}
        \setlength\itemsep{0.8em} % Adjust the spacing between items
        \item \textbf{Configuration du scraping Hespress}
        \item \textbf{Monitoring système et performances}
        \item \textbf{Gestion des utilisateurs et permissions}
        \item \textbf{Gestion des logs et alertes}
    \end{itemize}
\end{frame}
